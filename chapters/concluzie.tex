\chapter{Concluzii şi Dezvoltării ulterioare}
\label{chapter:concluzii}
\section{Concluzii}

Această aplicaţie îşi propune să rezolve o problemă concretă, şi anume permite interconectarea mai dispozitivelor inteligente, dispozitive ce se conectează la rândul lor la diverse surse de date. În această lume emergentă, serviciile sunt din ce în ce mai specializate, satisfăcând în mod complet o nişă. Orice persoană poate achiziţiona câţiva senzori inteligenţi, însă, pentru a obţine valoare din acei senzori, datele trebuie colectate, procesate şi stocate. 

Soluţia propusă răspunde astfel atât la problema colectării datelor din surse variate, cât şi la problema transformării acelor date în date utilizabile de alte sisteme, generând informaţii din date neprelucrate.

\section{Dezvoltării ulterioare}

Un aspect foarte important ce poate face scopul unei dezvoltării ulterioare este scalarea platformei într-o instantă cloud, ce poate fi folosită cu o arhitectura de tipul platform as a service (PaaS). Astfel, un utilizator poate externaliza serviciul de procesare a datelor, singura lui grijă fiind colectarea datelor în sistem.

Împreună cu dezvoltarea de mai sus, o alta direcţie este alinierea cât mai bună la conceptele din IoT (Internet of Things). Aceasta aliniere s-ar putea realiza prin transformarea blocului de intrare în "thing"-uri, cu proprietăţi şi servicii. Tot în această direcţie de dezvoltare s-ar putea include dezvoltarea de agenţi capabili să interfaţeze între protocoale de date proprietare, şi soluţia propusă. Spre exemplu, implementarea unui client MQTT ar permite conectivitatea către o întreagă gama de dispozitive care respectă acest standard.

În vederea uşurării dezvoltării aplicaţiilor pe aceasta platformă trebuie suplimentat numărul de blocuri de procesare implicite existente în sistem.