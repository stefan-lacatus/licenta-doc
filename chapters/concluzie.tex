\chapter{Concluzii şi Dezvoltării ulterioare}
\label{chapter:concluzii}
\section{Concluzii}

Aceasta aplicaţie satisface cerinţa lumii moderne în care avem din ce în ce mai multe dispozitive inteligente, capabile sa se conecteze la diverse surse de date. În aceasta lume emergenta, serviciile sunt din ce în ce mai specializate, satisfăcând în mod complet o nişă. Orice persoana poate achiziţiona câţiva senzori inteligenţi, însă, pentru a obţine valoare din acei senzori, datele trebuie colectate, procesate şi stocate. 

Soluţia propusă răspunde astfel atât la problema colectării datelor din surse variate, cat şi la problema transformării acelor date în date utilizabile de alte sisteme, generând informaţii din date neprelucrate.

\section{Dezvoltării ulterioare}

Un aspect foarte important ce poate face scopul unei dezvoltării ulterioare este scalarea platformei într-o instantă cloud, ce poate fi folosita cu o arhitectura de tipul platform as a service (PaaS). Astfel, un utilizator poate externaliza serviciul de procesare a datelor, singura lui grijă fiind aducerea tuturor acelor date în sistem.

Împreună cu dezvoltarea de mai sus, o alta direcţie este alinierea cât mai bună cu conceptele din IoT (Internetul tuturor obiectelor). Aceasta aliniere s-ar putea realiza prin transformarea blocului de intrare în "thing"-uri, cu proprietăţi şi servicii. Tot în aceasta direcţie de dezvoltare s-ar putea include dezvoltarea de agenţi capabili sa interfaţeze între protocoale de date proprietare, şi soluţia propusă. Spre exemplu, implementarea unui client MQTT ar permite conectivitatea către o întreagă gama de dispozitive care respecta acest standard.

In vederea uşurării dezvoltării aplicaţiilor pe aceasta platforma trebuie suplimentat numărul de blocuri de procesare implicite existente în sistem.