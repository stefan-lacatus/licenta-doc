\chapter{Introducere}

Într-o lume emergenta a dispozitivelor inteligente şi cu puternici capabilităţi de conectare, problema folosiri în mod corect şi eficient a datelor devine din ce mai importantă;. Mai multi analişti în domeniu estimează creşteri ameţitoare ale numărului de dispozitive: $5*10^9$ pana în 2020, $10^{12}$ pana în 2035 \autocite{iotGrowth} . Deşi aceste estimări sunt foarte apropiate de certitudini, ce este însa neclar, este modul în care aceste dispozitive vor fi folosite pentru a realiza aplicaţii conectate, care înglobează uniform date din sisteme complet diferite.

Problema poate fi generalizată la următoarele trei concepte:
\begin{itemize}
	\item  \textbf{Colectarea de date:} Tot mai multe dispozitive implică tot mai multe date, în formate variate. Aceste date trebuie stocate în mod eficient folosind instrumente de stocare scalabile. 
	\item \textbf{Prelucrarea datelor}: Date din surse variate trebuie combinate pentru a obţine informaţii. Aici sunt incluse atât datele stocate local, cât şi date disponibile din alte sisteme şi servicii ce trebuie interogate. Din aceste date se urmăreşte generarea de date noi, cu valoare şi semnificaţie mai mare.
	\item \textbf{Conectarea}: Datele procesate trebuie făcute accesibile pentru sisteme exterioare.
\end{itemize}

Problema aceasta se întâlneşte atât în mediile industriale, cat şi în aplicaţiile de mici dimensiuni. 
In ultimii 5 ani, a avut loc o noua revoluţie industrială, de la industria clasică la o paradigmă nouă a proceselor de manufactură, prin digitalizarea completă a întregului proces de fabricaţie în care toate părţile acestuia sunt conectate central \autocite{deloitteReport}. 
Tradiţional, această interconectare s-a făcut prin intermediul automatelor programabile, conectate la un sistem de tip SCADA. Această structură are totuşi dezavantajul ca este statica, şi nu valorifica în totalitate existenta unor senzori inteligenţi, iar fiecare automat trebuie programat individual pentru un singur scop, datele achiziţionate fiind greu de centralizat.

Pe de altă parte, pentru aplicaţiile din mediul privat, sistemele care centralizează şi procesează datele sunt fie foarte scumpe, restricţionând utilizatorul la senzori şi dispozitive produse de un singur producător, fie sunt greu de configurat, necesitând multe echipamente separate de procesare. Spre exemplu, o automatizare a casei folosind echipamente Siemens costa în jur de 15000\euro, dispozitivele de procesare rereprezentând jumătate din acest preţ. Pe de alta parte, pachete software open-source, precum OpenHAB \autocite{openHab} rezolva un subset al problemei, uşurând integrarea echipamentelor compatibile, însă nu sunt destul de generice, având aplicabilitate într-un singur domeniu.

Soluţia propusă doreşte să răspundă la problema aplicaţilor de dimensiuni mici şi mijlocii, în special din mediul privat care, propunând o alternativă la achiziţionarea de PLC-uri sau alte dispozitive complicate pentru stocarea şi procesarea datelor. Acesta unifica achiziţia, procesarea, stocarea şi distribuţia datelor într-o singura aplicaţie ce poate fi instalată atât local, cat şi pe o instantă cloud, permiţând accesul din orice locaţie. Un alt avantaj al unificării tuturor acestor funcţionalităţi într-o singura aplicaţie, este prezentarea unui interfeţe standard pentru implementarea proceselor de procesare. Astfel, se evita unul din cele mai mari avantaje ale automatelor programabile, la care, fiecare producător are propriile standarde de implementare a aplicaţiilor.

Una din cerinţele iniţiale pe care aplicaţia a urmărit să le respecte este aspectul generic a datelor. Deşi majoritatea aplicaţiilor vor fi în procesarea numerică, acesta este doar unul din mai multe tipuri de date. Acest aspect generic permite folosirea soluţiei ca o magistrală de procesare a datelor în medii business, care transformă date complexe produse de un sistem în date ce pot fi acceptate de alt sistem, menţinând un istoric în tot acest timp.

Acesta lucrare urmăreşte prezentarea implementării acestei soluţii. \Cref{chapter:arhitecture} prezintă arhitectura generală, descriind entităţile folosite şi modul în care acestea interacţionează. \Cref{chapter:interfata} ilustrează interfaţa vizuală pe care aplicaţia o pune la dispoziţia utilizatorului, iar \cref{chapter:implemetare} descrie în detaliu implementarea. Ultimele două capitole, \ref{chapter:studiuCaz} şi \ref{chapter:concluzii}, prezintă un caz de utilizare al aplicaţiei respectiv câteva concluzii şi dezvoltării ce ar putea îmbunătăţi soluţia propusă.